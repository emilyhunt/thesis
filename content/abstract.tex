% !TEX root = ../my-thesis.tex
%

\pdfbookmark[0]{Abstract}{Abstract}
\addchap*{Abstract}
\label{sec:abstract}

% --------------- ENGLISH
% 5 words by default

For over a century, open clusters have been a key tool for understanding stellar and galactic evolution. Now, thanks to groundbreaking new astrometric and photometric data from the European Space Agency's \gaia\ satellite, it is possible to study open clusters to a never before seen level of accuracy and precision. In this thesis, I develop and apply new methodologies to improve the census of open clusters using \gaia\ data, with a focus on using modern, efficient, and statistically rigorous techniques to maximise the reliability and usefulness of the open cluster census, despite the many challenges of working with the billion-star dataset of \gaia. Firstly, I conducted a comparative study of clustering algorithms for retrieving open clusters blindly from \gaia\ data, finding that a previously untrialed algorithm (HDBSCAN) is the most sensitive algorithm for open cluster recovery. Next, using this methodology, I used \gaia\ DR3 data to create the largest homogeneous catalogue of open clusters to date, recovering a total of 7167 clusters -- 2387 of which are candidate new objects. I developed an approximate Bayesian neural network for classifying the reliability of the colour-magnitude diagrams of the clusters in the census, and used a modification of this network to infer parameters such as the age and extinction of these clusters. Finally, since many of the objects in the census appeared more compatible with moving groups, I measured accurate masses, Jacobi radii, and velocity dispersions for these clusters, creating the largest catalogues of these parameters for open clusters to date. Using these parameters, I show that no more than 5619 of the clusters in the census are compatible with bound open clusters. Additionally, I use these mass estimates to derive an approximate completeness estimate for the \gaia\ DR3 open cluster census, finding that the approximate 100\% completeness limit depends strongly on cluster mass. The results of this thesis show that it is possible to reliably create a catalogue of open clusters with a single blind search, in addition to measuring parameters for these objects. The methods developed in this thesis will be applicable to future data releases from \gaia\ and other sources.




% --------------- GERMAN
\newpage
%\vspace*{20mm}

{\usekomafont{chapter}Zusammenfassung}
\label{sec:abstract-diff}

\todo{german abstract}

% !TEX root = ../my-thesis.tex
%
\chapter{Conclusion}
\label{sec:conclusion}

\cleanchapterquote{No observational problem will not be solved by more data.}{Vera Rubin}{}


Finally, in this chapter, I conclude my thesis with a review of various aspects of my work. Firstly, I briefly summarise the key results of this thesis. Secondly, I provide some commentary on how my thesis contributes to solving the five broad problems with the current census of OCs that I identified in Sect.~\ref{sec:intro:aims:issues}, as well as identifying some areas in which these problems could be addressed further with future work.


\section{Summary of the results of this work}

In this thesis, I worked to improve the methods for detecting, cataloguing, and analysing OCs using the astrometric and photometric data of \gaia. Beyond simply showing that these methodologies are effective, I used these methods to create the largest homogeneous catalogue of OCs to date. With a focus on rigorous, statistical approaches, I quantified the reliability of all objects in the catalogue and derived parameters for them, in addition to reporting over two thousand new star clusters and showing that over a thousand OCs reported before \gaia\ are unlikely to be real. Finally, I derived the largest ever sample of cluster masses and Jacobi radii, using them to distinguish between bound OCs and unbound MGs and derive an approximate completeness estimate for the \gaia\ DR3 OC census.

After reviewing the current status of the literature surrounding OCs in the introduction to this thesis, in the first scientific chapter (Sect.~\ref{sec:comparison}), I reviewed and trialled a number of different methods to recover OCs from \gaia\ data. Since the release of \gaia\ DR1, automated clustering algorithms have been used to tremendous success to detect new OCs in \gaia\ data or to derive membership lists for existing clusters \citep[e.g.][]{castro-ginard_new_2018,castro-ginard_hunting_2019,castro-ginard_hunting_2020,castro-ginard_hunting_2022,liu_catalog_2019,cantat-gaudin_characterising_2018,cantat-gaudin_clusters_2020,cantat-gaudin_gaia_2019,he_catalogue_2021,he_new_2022,hao_sixteen_2020,jaehnig_membership_2021}. However, despite the widespread use of numerous algorithms in dozens of different works, the different methodologies used in the literature had never been compared side by side for application on OCs.

In Sect.~\ref{sec:comparison}, I compared three different algorithms for OC recovery -- DBSCAN, HDBSCAN, and Gaussian mixture models. Using the results of a practical test of the three algorithms applied to \gaia\ DR2 data, I found that no algorithm was without flaws. However, HDBSCAN was clearly the most sensitive for OC recovery -- despite not having been used by any authors at the time for detecting OCs, with almost all works preferring to use DBSCAN or Gaussian mixture models. To mitigate issues with HDBSCAN reporting a high number of false positive clusters, I developed a computationally efficient statistical density test to postprocess standard OC clustering results and derive an astrometric S/N for candidate clusters, which acts as a description of the quality of detected objects. Finally, I was able to detect 41 new OCs within the data in this work. 

Having shown that HDBSCAN is the most sensitive method for OC recovery, in Sect.~\ref{sec:census}, I used \gaia\ DR3 data to conduct the largest all-sky blind search for OCs to date, searching for OCs amongst a total of 729~million stars. By conducting a single search with HDBSCAN, I was able to recover a large fraction of all reported OCs, including virtually all reliable objects from the previous largest catalogue of OCs of \cite{cantat-gaudin_clusters_2020}. To further classify objects by reliability, I used an approximately Bayesian neural network to classify clusters based on the compatibility of their CMD with a single population of stars. Simple modifications to this network architecture also allowed for the inference of ages, extinctions, and photometric distances to the clusters within the catalogue.

In total, the catalogue contains 7167 clusters, 2387 of which are candidate new objects. 4105 of these objects are of high quality, with 739 high-quality objects being new. This represents the largest homogeneous catalogue of OCs and OC candidates in the \gaia\ era. The catalogue was the first time that many recently reported OCs had been recovered by an independent work, allowing for the confirmation of over 2000 objects as being likely to be real. In addition, owing to the scope of the search, I was able to show that 1152 clusters from the pre-\gaia\ catalogue of \cite{kharchenko_global_2013} that are as yet undetected in \gaia\ data are unlikely to be real, representing the largest analysis to date of which clusters detected before \gaia\ are or are not real.

I showed that many of the clusters in the catalogue appear more compatible with unbound MGs, and that basic empirical cuts on mean cluster parameters are insufficient to remove MGs from the catalogue. Motivated by a need to differentiate between bound and unbound clusters more accurately, in Sect.~\ref{sec:dynamics}, I present preliminary results into a study to measure the masses, velocity dispersions, and Jacobi radii of the largest ever sample of clusters, with the aim to use these parameters to discern between OCs and MGs.

I derived these parameters for 6974 clusters, representing a catalogue of accurate cluster masses more than 70 times larger than the largest other catalogue of cluster masses using \gaia\ data. I showed that 82\% of the clusters from Sect.~\ref{sec:census} are compatible with bound objects, dropping to just 16\% of those within 250~pc where the Sect.~\ref{sec:census} catalogue appeared to be dominated by moving groups. I showed that cluster masses and Jacobi radii are an effective way to differentiate between MGs and OCs, refining the observational definition of OCs into a more theoretically grounded definition.

On the other hand, I found that velocity dispersions measured using proper motions are an unreliable way to probe the boundness of an OC or OC candidate. I conducted simulations that show that unresolved and resolved binary stars may dominate the velocity dispersion of most clusters, and especially those of lower masses. This is at odds with many recent results that have suggested that OCs are supervirial \citep[e.g.][]{bravi_gaia-eso_2018,kuhn_kinematics_2019,pang_3d_2021}, which typically do not correct for or address the possibility of binary star contamination of proper motion measurements.

In total, I was able to show that there are 5619 probable OCs in my catalogue, 3515 of which are of high quality. Using these samples of clusters, I derive the largest ever mass-dependent completeness estimate for the OC census. I found that the completeness is well-described by a log-linear relation in mass until around $10^3$~\MSun, after which the 100\% completeness limit for all higher masses of cluster in \gaia\ DR3 is around 2700~pc.


\section{Outlook on the future of OC science}

At the beginning of this thesis, I painted a bright view of the scientific potential of OCs in the Milky Way. For over a century, these objects have played a key role in furthering our understanding of stellar and galactic evolution. Thanks to \gaia, Milky Way astronomy is currently undergoing a renaissance: but the methodologies and catalogues used to analyse OCs must keep pace with the incredible data that \gaia\ is providing. With this thesis, I have attempted to tackle the issues with the OC census I outlined in Sect.~\ref{sec:intro:aims:issues}; to conclude this thesis, I think it is interesting to discuss how I have tried to contribute to solving these problems, and how these problems could be further addressed in the future.


\subsubsection{Problem 1. The methods used to detect open clusters (and their biases)}

The overview of algorithms presented in Sect.~\ref{sec:comparison} provided a number of useful things to the OC community. By showing that HDBSCAN is the most sensitive algorithm, despite having been used minimally in the literature, this work showed that current approaches were inadequate to effectively detect OCs. The catalogue of OCs in Sect.~\ref{sec:census} confirmed this finding and showed that HDBSCAN can be used to conduct large blind searches and create a single, homogeneous catalogue. While I think that this thesis demonstrates that HDBSCAN is the best \emph{currently available} methodology for OC retrieval, there are still many things that it leaves to be desired.

During the preparation of the catalogue in Sect.~\ref{sec:census}, I encountered many difficulties due to computational limitations. Chief amongst those was that the main loop of HDBSCAN is single threaded, greatly limiting how large of a dataset the algorithm can be run on, and meaning that the \gaia\ dataset had to be tiled into over 12\,000 separate regions that required a painstaking merging process to recombine. A faster, parallelised version of HDBSCAN is currently under development, and could offer a significantly more straightforward way to construct a blind search catalogue\footnote{\url{https://github.com/TutteInstitute/fast_hdbscan}}. 

In addition, I think it is quite unsatisfactory that use of HDBSCAN requires postprocessing initial results with a density test to remove false positives. Algorithms that report mostly incorrect results without further processing open the door to other issues. The papers of \cite{kounkel_untangling_2019} and \cite{kounkel_untangling_2020} are one such example of an incorrect result derived using clustering algorithms being published. In those works, the authors used HDBSCAN to report thousands of `groups' and `strings' within 3~kpc. The immense size and scale of these objects was a result with a certain `wow factor,' which led to the papers being highlighted in multiple press releases\footnote{\url{https://www.esa.int/Science_Exploration/Space_Science/Gaia/Gaia_untangles_the_starry_strings_of_the_Milky_Way}}. Yet despite also using HDBSCAN, I was only able to recover 18\% of the objects they reported in Sect.~\ref{sec:census} (most of which are groups that are OCs), and analysis by \cite{zucker_disconnecting_dots_2022} showed that their strings and groups are incompatible with physically associated objects. If HDBSCAN did not require so much extra vigilance from its users, then this set of now-contested results would have never been published.

Finally, although advantageous for reasons of speed, some detail is lost by the fact that HDBSCAN does not incorporate error information during its clustering analysis. Particularly for distant objects, it is likely that some member stars are missed relative to methods such as UPMASK (see Sect.~\ref{c3:sec:results-overall}). There is currently no method that can include error information while also being feasible to run on a large input list of sources (UPMASK has an extremely slow runtime compared to HDBSCAN). The golden bullet would be a clustering algorithm that can include error information while also having the $\mathcal{O}(n\log n)$ runtime complexity that makes methods such as HDBSCAN feasible to run on such large datasets.

Given the pace of modern machine learning literature, I would not be surprised if an even better clustering algorithm for OC recovery emerges by the end of the decade. Astronomers should continue to keep a close eye on the machine learning literature, and should also consider direct collaborations with computer scientists to attempt to improve existing approaches.


\subsubsection{Problem 2. The status of clusters discovered before \gaia}

In comparison, Problem~2 is more of a closed case. In Sect.~\ref{sec:census}, I showed that at least 1000 objects from \cite{kharchenko_global_2013} should be detectable in \gaia\ data, but do not appear to be real objects. Works that still use pre-\gaia\ catalogues of OCs \citep[such as the catalogue of OC masses in][]{just_global_survey_2023} are likely to reach flawed conclusions, given that there are many objects in works such as \cite{kharchenko_global_2013} within a few kpc and with low reported extinctions that should be detectable in \gaia\ data but do not seem to exist.

However, there are some IR clusters from works such as \cite{kharchenko_global_2013} that \gaia\ will potentially never be able to detect. Particularly when peering deep into the galactic disk, extinction quickly terminates the ability to perform analysis of clusters within heavily reddened regions. There are some fascinating clusters within nebulae like the Carina nebula that \gaia\ is inherently limited in studying. For instance, Trumpler~14 is an extremely young cluster with an age of just $\sim0.4$~Myr, for which ground based photometric studies have identified as many as 2000 member stars \citep{sana_mad_view_2010}; using \gaia\ data, I identify just 96 member stars in Sect~\ref{sec:census}. 

The proposed \emph{GaiaNIR} mission \citep{hobbs_gaianir_combining_2016} could contribute massively to analysis of star clusters within reddened regions of the Milky Way. \emph{GaiaNIR} would make it possible to peer deeper into the galactic disk than ever before, in addition to allowing for new analysis of some IR clusters that \gaia\ has likely missed. This would be particularly helpful for astrometric studies of heavily reddened or obscured star formation regions such as the Carina nebula.


\subsubsection{Problem 3. The status of clusters discovered with \gaia}

In Sect.~\ref{sec:census}, I was able to confirm that over 2000 clusters discovered using \gaia\ appear to be real objects that can be re-detected independently. Thanks to \gaia\ data, the OC census is thousands of objects larger than it was before in works such as \cite{kharchenko_global_2013}, and these results have now been demonstrated to remain consistent and reliable across multiple \gaia\ data releases and with at least one independent confirmation.

However, there are many objects I was not able to find in Sect.~\ref{sec:census}. For instance, despite the generally high reliability of UBC clusters reported using \gaia\ data in works since \cite{castro-ginard_new_2018}, I was only able to recover 89.2\% of clusters reported using \gaia\ DR2 in \cite{castro-ginard_hunting_2020} and 88.9\% of clusters reported using \gaia\ DR3 in \cite{castro-ginard_hunting_2022}. What is the status of the remaining missing objects?

For clusters reported in DR2 that cannot be recovered in DR3, it is likely that these objects are not real clusters. The jump from DR2 to DR3 resulted in clusters having astrometric S/N around $10\sigma$ higher. If these objects are real, they should stand out clearly in DR3. On the other hand, for the many clusters reported using DR3 that are not possible to recover \citep[e.g. 55.5\% of][]{he_unveiling_hidden_2022}, one could argue that HDBSCAN may have some insensitivity to some of those clusters -- particularly those at large distances. However, many of the unreported clusters from works such as \cite{he_unveiling_hidden_2022} are nearby and should be well within the range in which HDBSCAN is the most sensitive algorithm (Sect.~\ref{sec:comparison}), suggesting that many of the objects we do not detect in the catalogue in Sect.~\ref{sec:census} are unlikely to be real.

Future data releases will always be a good tool to determine more precisely which objects are or are not real. However, with so many objects still missing from the catalogue in Sect.~\ref{sec:census}, it would be interesting to conduct a follow-up study of the missing objects, focusing on studying a small sample of missing clusters in greater depth. It may be that some objects (such as missing UBC clusters) can be recovered using another methodology, or it may be that many simply do not exist. The study of \cite{piatti_assessing_2023} is one such example of an insightful detailed study of a smaller sample of objects.


\subsubsection{Problem 4. The completeness of the open cluster census}

The completeness of the OC census has been improved substantially by \gaia\ data, of which this thesis has made multiple contributions. Firstly, by reporting thousands of new objects, I have helped to expand the census of OCs. This includes some objects that I am surprised were missed until now, such as HSC~2384 -- which is a cluster of around 200~\MSun\ that is only around 550~pc away, having been obscured previously by IC~2602. In particular, given that almost all works to detect new OCs have used DBSCAN until now, which is less sensitive to nearby objects (Sect.~\ref{sec:comparison}), my new clusters help to plug a gap in clusters that may have been missed by other methodologies.

Secondly, by deriving the largest homogeneous catalogue of OCs to date in Sect.~\ref{sec:census}, in addition to complementing this with parameters such as cluster age and mass, I hope I have made a catalogue that will remain useful to many in the literature for many years to come -- particularly due to the scope of the search and the sheer number of clusters catalogued accurately within one work. It has previously been claimed that it is not possible to construct such a catalogue with only one algorithm \citep{cantat-gaudin_clusters_2020}. By showing that this claim is incorrect, my catalogue presents a new way to construct a census of OCs given any dataset. I hope that this will pave the way for future studies using \gaia\ DR4 and DR5 (as well as whatever is a successor to \gaia!). Single homogeneous catalogues including both existing and new objects can be constructed with one sensitive algorithm.

The clear advantage of a single homogeneous catalogue is that the completeness of the catalogue is significantly easier to derive. In Sect.~\ref{sec:dynamics:discussion}, I show briefly that the cluster masses I have derived are a good tracer of the completeness of our OC catalogue. An excellent follow-up to this work would be to derive a completeness estimate depending on more variables, such as cluster mass, age, distance, extinction, and position. However, such work would be non-trivial due to the often correlated nature of OC locations within the galactic disk \citep{anders_milky_2020}. However, such work would be highly interesting, as a robust completeness estimate would allow for a precise determination of the Milky Way's cluster age function and cluster mass function.

Finally, I expect that future \gaia\ data releases and potential future instruments will continue to make major contributions to the OC census. There may be as many as $10^5$ OCs in the entire Milky Way \citep{dias_new_2002}, of which around 5000 are currently known. The potential to find new OCs at ever-greater distances will be effectively limitless for at least the coming century. In turn, new OC discoveries will allow for OCs to continue to be used to trace interesting properties of the structure of the Milky Way out to ever-increasing distances. 


\subsubsection{Problem 5. The observational definition of open clusters}

The last problem I attempted to solve in this thesis was that of how best to define OCs, which I worked on throughout the duration of my PhD. Compared to when I began my PhD, I think that it is now significantly easier to define OCs observationally. I used the statistical density test from Sect.~\ref{sec:comparison} to quantify how dense a cluster is, the CMD classifier from Sect.~\ref{sec:census} to quantify how compatible a cluster is with a single population of stars, and the Jacobi radii of clusters from Sect.~\ref{sec:dynamics} to measure the boundness of star clusters. In addition, all of these criteria are probabilistic, meaning that edge-case clusters can be adequately handled and are not simply rejected.

In particular, I was thoroughly impressed by how effective the Jacobi radius method was, given that it was derived from a study of just ten clusters near to the Sun \citep{meingast_extended_2021}. I think that there is an untapped goldmine of potential in measuring star cluster masses with \gaia\ data. This is particularly because of recent strides in determining \gaia's selection function \citep{cantat-gaudin_empirical_model_2023,castro-ginard_estimating_selection_2023}, which allows for cluster masses to be measured to a higher level of accuracy. It goes without saying that the work from Sect.~\ref{sec:dynamics} should be published as soon as possible, particularly given the role it will play in dramatically improving the usability of the results in the main catalogue of clusters. In addition, it may also be worth publishing an open source library that is able to calculate properties of clusters such as their masses and Jacobi radii, which would make it easier for other works to use the methodology I have created.

With the results of this thesis in mind, I feel that Problem 5 can be answered well by the following four definitions:

\begin{itemize}
    \item At least ten member stars (distinguishing it from a multiple star system.)
    \item A statistically significant overdensity within a dataset such as \gaia.
    \item A colour-magnitude diagram compatible with a single population of stars.
    \item A valid Jacobi radius (implying that the cluster is bound.)
\end{itemize}

At this time, I would not recommend including velocity dispersions within these criteria. Given that many OCs appear to have tidal tails \citep{tarricq_structural_2022}, this implies that OCs are dynamically warm enough that they are at least \emph{slightly} supervirial. It is difficult to define a reasonable cut on the virial ratio equation (Eqn.~\ref{eqn:intro:virial_ratio}), and it can be extremely challenging to adequately clean a cluster membership list of stars that are no longer bound to the cluster. 

This cut would be even more challenging to set since some clusters are currently undergoing strong disruption, such as the Hyades or Platais~9. The clusters are probably quite strongly supervirial, with many stars within each cluster being unbound. But I do not feel it would make sense to say that the Hyades is no longer an OC simply because it only has a few tens of Myr of life remaining \citep{oh_kinematic_modelling_2020}; rather, it is a \emph{disrupting} OC. This mirrors some other principles of definitions in astronomy: for instance, even though red giant stars are quite different to stars on the main sequence, they are still stars. 

Velocity dispersions are probably most useful as probes of the dynamical state of OCs in different stages of their lives, but do not appear particularly practical or useful for distinguishing between OCs and MGs. In fact, the investigations into binary star contamination in Sect.~\ref{sec:dynamics} suggest that velocity dispersions of OCs in \gaia\ DR3 are of limited use, since it is easy to create scenarios where a cluster has just a few hundred m\,s\textsuperscript{-1} of binary star dispersion contamination, which results in an incorrect measurement of the cluster's dispersion or virial ratio by at least a factor of two.

Improved astrometry in \gaia\ DR4 could be extremely helpful in reducing binary star contamination. The main catalogue of \gaia\ DR4 is expected to contain numerous astrometric binary candidates, which could simply be removed from measurements of cluster dynamics. It may also be possible to use time series astrometry that should be released with \gaia\ DR4 directly. Having assigned stars as members of a real physical cluster, strict priors could be placed on the star's parallax and proper motion. These priors could help to further refine an astrometric fit, increasing the ease with which astrometric binaries can be identified within clusters. 

% A question that I asked myself throughout my work on the catalogue of clusters was how much difference there really fundamentally is between different types of star cluster. The smallest, most heavily disrupted OCs (such as Platais~9) barely have a core, and are heavily reminiscent of MGs except for a small bound core of stars that will remain bound together for only a few million years. The fact that HDBSCAN detects so many MGs has been a problem that has taken my entire PhD to solve, culminating in Sect.~\ref{sec:dynamics} of this thesis. But it begs the question: is it even a bad thing to have MGs and OCs in the same catalogue? Given that dissolved OCs should be detectable as MGs for a few tens of Myr \citep{portegies_zwart_young_2010}, MGs are an important way to study the afterlife of OCs. On the other hand, stars can form in an unbound MG that stays together for a short duration (e.g. Monoceros~R2, Fig.~\ref{fig:intro:definition:comparison}); but at least in terms of age and chemical homogeneity, an MG that never derived from a bound cluster is more or less just as good as an OC to study stellar evolution (and is an important part of the stellar lifecycle regardless). Maybe \gaia's wealth of MGs are more of a blessing than a curse: these hundreds of objects probably deserve just as much study as OCs; and if the methods used to detect OCs can detect MGs at the same time, then why not do so?


\section{Final remarks}

As I reach the last sentences of this thesis, I want to emphasise just how important it is that work to improve the census of OCs does not end here. New data, methodologies, theories, and computational resources will inevitably become available. 

With \gaia\ DR4 slated for release in around 2.5 years at the time of writing \citep[no sooner than the end of 2025,][]{gaia_collaboration_gaia_2022}, and \gaia\ DR5 coming around the end of the decade, there are multiple even larger data releases on the horizon. DR4 will provide proper motions with doubled accuracy, significantly more identified astrometric binary stars, and most likely improved processing that reduces the number of bad sources.

Since the times of the 1700s when reflecting telescopes gained widespread use, the effort to improve catalogues of objects in the night sky has been never-ending. I expect that this long-held tradition will continue with future \gaia\ data releases, as well as potential \gaia\ follow-ups such as \emph{GaiaNIR}. After all, in just five years since the release of \gaia\ DR2, the OC census has been changed beyond recognition by the incredible data available to the current generation of Milky Way astronomers. I am excited to see where the open clusters take us next.




% SCIENCE REVIEW
% - step by step -> what did I do?
% - relate back to identified problems with the OC census

% HOW RELEVANT
% - how will people use my results?
% - how does it help others?

% TAKE HOME MESSAGE
% - gaia, lsst, gaianir is gonna keep on giving
% - improving the OC census will be a never-ending task, but the payoff is pretty gorgeous!


% !TEX root = ../my-thesis.tex
%
\chapter{Conclusion}
\label{sec:conclusion}

\cleanchapterquote{No observational problem will not be solved by more data.}{Vera Rubin}{}

At the beginning of this thesis, I painted a bright view of the scientific potential of OCs in the Milky Way. For over a century, these objects have played a key role in furthering our understanding of stellar and galactic evolution. Thanks to \gaia, Milky Way astronomy is currently undergoing a renaissance: but the methodologies and catalogues used to analyse OCs must keep pace with the incredible data that \gaia\ is providing to modern astronomy.

In this thesis, I worked to improve the methods for detecting, cataloguing, and analysing OCs using the astrometric and photometric data of \gaia -- targeting five broad problems with the current census of OCs. Beyond simply proving that these methodologies are effective, I used these methods to create the largest homogeneous catalogue of OCs to date. With a focus on rigorous, statistical approaches, I quantified the reliability of all objects in the catalogue and derived parameters for them, in addition to reporting over two thousand new star clusters and showing that over a thousand OCs reported before \gaia\ are unlikely to be real. Finally, I derived accurate cluster masses and Jacobi radii for the largest ever sample of clusters for such parameters, using them to distinguish accurately between bound OCs and unbound MGs and derive an approximate completeness estimate for the \gaia\ DR3 OC census.

After reviewing the current status of the literature surrounding OCs in the introduction to this thesis, in the first scientific chapter (Sect.~\ref{sec:comparison}), I reviewed and trialed a number of different methods to recover OCs from \gaia\ data. Since the release of \gaia\ DR1, automated clustering algorithms have been used to tremendous success to detect new OCs in \gaia\ data or to derive membership lists for existing clusters \citep[e.g.][]{castro-ginard_new_2018,castro-ginard_hunting_2019,castro-ginard_hunting_2020,castro-ginard_hunting_2022,liu_catalog_2019,cantat-gaudin_characterising_2018,cantat-gaudin_clusters_2020,cantat-gaudin_gaia_2019,he_catalogue_2021,he_new_2022,hao_sixteen_2020,jaehnig_membership_2021}. However, despite the widespread use of numerous algorithms in dozens of different works, the different methodologies used in the literature had never been compared side by side for application on OCs.

In Sect.~\ref{sec:comparison}, I compared three different algorithms for OC recovery -- DBSCAN, HDBSCAN, and Gaussian mixture models. Using the results of a practical test of the three algorithms applied to \gaia\ DR2 data, I found that no algorithm was without flaws. However, HDBSCAN was clearly the most sensitive for OC recovery -- despite not having been used by any authors at the time for detecting OCs, with almost all works preferring to use DBSCAN or Gaussian mixture models. To mitigate issues with HDBSCAN reporting a high number of false positive clusters, I developed a computationally efficient statistical density test to postprocess standard OC clustering results and derive an astrometric S/N for candidate clusters, which acts as a description of the quality of detected objects. Finally, I was able to detect 41 new OCs within the data in this work. 

Having proven HDBSCAN as the most sensitive method for OC recovery, in Sect.~\ref{sec:census}, I used \gaia\ DR3 data to conduct the largest all-sky blind search for OCs to date, searching for OCs amongst a total of 729~million stars. By conducting a single search with HDBSCAN, I was able to recover a large fraction of all reported OCs, including virtually all reliable objects from the previous largest catalogue of OCs of \cite{cantat-gaudin_clusters_2020}. To further classify objects by reliability, I used a supervised approximately Bayesian neural network to classify clusters based on the compatibility of their CMD with a single population of stars. Simple modifications to this network architecture also allowed for the inference of ages, extinctions, and photometric distances to the clusters within the catalogue.

In total, the catalogue contains 7167 clusters, 2387 of which are new. 4105 of these objects are of high quality, with 739 high-quality objects being new. This represents the largest homogeneous catalogue of OCs and OC candidates in the \gaia\ era, being more than three times the size of that of the previous largest catalogue, \cite{cantat-gaudin_clusters_2020}. The catalogue was the first time that many recently reported OCs had been recovered by an independent work, allowing for the confirmation of over 2000 objects as being likely to be real. In addition, owing to the scope of the search, I was able to show that 1152 clusters from the pre-\gaia\ catalogue of \cite{kharchenko_global_2013} that are as yet undetected in \gaia\ data are unlikely to be real, representing a large fraction of the clusters as-yet undetected in \gaia.

I showed that many of the clusters in the catalogue appear more compatible with unbound MGs, and that basic empirical cuts on mean cluster parameters are insufficient to remove MGs from the catalogue. Hence, in Sect.~\ref{sec:dynamics}, I present preliminary results into a study to measure the masses, velocity dispersions, and Jacobi radii of the largest ever sample of clusters and use them to discern between OCs and MGs, refining the observational definition of OCs into a more theoretically grounded definition.

I derived these parameters for 6974 clusters, repesenting a catalogue of accurate cluster masses more than 70 times larger than the largest other catalogue of cluster masses in \gaia. I showed that 82\% of the clusters from Sect.~\ref{sec:census} are compatible with bound objects, dropping to just 16\% of those within 250~pc where the Sect.~\ref{sec:census} catalogue was dominated by moving groups. I showed that cluster masses and Jacobi radii are an effective way to differentiate between MGs and OCs.

On the other hand, I found that velocity dispersions measured using proper motions are an unreliable way to probe the `boundness' of an OC or OC candidate. I conducted simulations that show that resolved and unresolved binary stars may dominate the velocity dispersion of most clusters, and especially those of lower masses. This is at odds with many recent results that have suggested that OCs are supervirial which often do not correct for or address the possibility of binary star contamination \citep[e.g.][]{bravi_gaia-eso_2018,kuhn_kinematics_2019,pang_3d_2021}.

Nevertheless, I was able to show that there are 5619 probable OCs in my catalogue, 3515 of which are of high quality. Using these samples of clusters, I derive the largest ever mass-dependent completeness estimate for the OC census, finding that the completeness is well-described by a log-linear relation until around 2700~pc, which appears to be a the highest possible 100\% completeness limit for all masses of clusters.

With this thesis, I have attempted to tackle the issues with the OC census I outlined in Sect.~\ref{sec:intro:aims:issues}. By providing an overview of algorithms, I have provided important analysis and information to the community, outlining the choices one could make in choosing an algorithm for OC recovery. Since its publication, my comparison of clustering algorithms has been cited by a number of works, with works such as \cite{dellacroce_ongoing_hierarchical_2023a} adopting HDBSCAN as well as the density test I designed for their analysis. 

The census of star clusters from Sect.~\ref{sec:census} should remain useful to many in the literature for many years to come, particularly owing to the scope of the search and the sheer number of clusters catalogued accurately within one work. It has previously been claimed that it is not possible to construct such a large catalogue with only one algorithm \citep{cantat-gaudin_clusters_2020}. By proving this claim to not be true, my catalogue presents a new way to construct a census of OCs given any dataset. I hope that this will pave the way for future studies using \gaia\ DR4 and DR5 (as well as whatever is a successor to \gaia!), in which single homogeneous catalogues can be constructed with one sensitive algorithm.

Finally, given issues surrounding MGs in the census, the final chapter of this thesis (Sect.~\ref{sec:dynamics}) provides many results that will greatly improve the existing census as well as existing definitions of OCs. By deriving cluster masses and Jacobi radii for such a large sample, I show that a precise definition of OCs can be applied across the entire galactic disk. This work should be published as soon as possible, particularly given the role it will play in dramatically improving the usability of the results in the main catalogue of clusters.

With \gaia\ DR4 slated for release in around 2.5 years at the time of writing \citep[no sooner than the end of 2025,][]{gaia_collaboration_gaia_2022}, and \gaia\ DR5 coming around the end of the decade, there are multiple even larger data releases on the horizon. DR4 will provide proper motions with doubled accuracy, significantly more astrometric binary stars, and most likely improved processing that reduces the number of bad sources. An OC census constructed from a billion input sources should probably be a target to complete for DR4.

During the preparation of the catalogue in Sect.~\ref{sec:census}, I encountered so many difficulties due to computational limitations. Chief amongst those was that the main loop of HDBSCAN is single threaded, greatly limiting how large of a dataset the algorithm can be ran on, and meaning that the \gaia\ dataset had to be tiled into over 12\,000 separate regions that required a painstaking merging process to recombine. A faster, parallelised version of HDBSCAN is currently under development, and could offer a significantly more straightforward way to construct a blind search catalogue\footnote{\url{https://github.com/TutteInstitute/fast_hdbscan}}. In fact, given the pace of modern machine learning literature, I would not be surprised if an even better clustering algorithm emerges by the end of the decade. Astronomers should continue to keep a close eye on the machine learning literature.

A question that I asked myself throughout my work on the catalogue of clusters was how much difference there really fundamentally is between different types of star cluster. The smallest, most heavily disrupted OCs (such as Platais~9) barely have a core, and are heavily reminiscent of MGs in all but a small Jacobi radius of stars that will remain bound together for a few more million years. The fact that HDBSCAN detects so many MGs has been a problem that has taken my entire PhD to solve, culminating in Sect.~\ref{sec:dynamics} of this thesis. But it begs the question: is it even a bad thing to have MGs and OCs in the same catalogue? Given that dissolved OCs should form detectable MGs for a few tens of Myr \citep{portegies_zwart_young_2010}, MGs are an important way to study the afterlife of OCs. On the other hand, stars can form in an unbound MG that stays together for a short duration (e.g. Monoceros~R2, Fig.~\ref{fig:intro:definition:comparison}); but at least in terms of age and chemical homogeneity, an MG is more or less just as good as an OC to study stellar evolution (and is in fact an important part of the stellar lifecycle regardless). Maybe \gaia's wealth of MGs are more of a blessing than a curse: these hundreds (and eventually thousands, I'm sure) of objects probably deserve just as much study as OCs -- and if the methods used to detect OCs can detect MGs at the same time, then why not do so?

Finally, as I reach the last sentences of this thesis, I want to emphasise just how important it is that work to improve the census of OCs does not end on the other side of this page. New data, methodologies, theories, computational resources, and more will always be available. Since the times of the 1700s when the refracting telescope gained widespread use, the effort to improve the census of OCs has been neverending. I expect that future \gaia\ data releases as well as potential \gaia\ follow-ups such as \emph{GaiaNIR} will continue this long-held tradition. After all, in just five years since the release of \gaia\ DR2, the OC census has been changed beyond recognition by the incredible strides in data and methodological quality available to modern Milky Way astronomy. I am excited to see where the open clusters take us next.




% SCIENCE REVIEW
% - step by step -> what did I do?
% - relate back to identified problems with the OC census

% HOW RELEVANT
% - how will people use my results?
% - how does it help others?

% TAKE HOME MESSAGE
% - gaia, lsst, gaianir is gonna keep on giving
% - improving the OC census will be a never-ending task, but the payoff is pretty gorgeous!


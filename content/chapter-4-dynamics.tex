% !TEX root = ../my-thesis.tex
%
\chapter{The masses and dynamics of star clusters in the Milky Way}
\label{sec:dynamics}

\cleanchapterquote{Things are only impossible until they're not.}{Jean-Luc Picard}{(2364)}

\authorship{The results presented in this chapter will be published in Hunt and Reffert (\emph{in prep.}). All calculations, figures, and writing in this chapter were conducted by myself.}

\todo{dynamics section}

% -------------------------------------
\section{Introduction}
\label{sec:dynamics:introduction}

Five years on since the release of \gaia\ DR2 \citep{brown_gaia_2018}, the census of open clusters (OCs) has been resoundingly overhauled \citep{cantat-gaudin_milky_2022}. Thousands of new objects have been discovered \citep[e.g.][]{liu_catalog_2019,castro-ginard_hunting_2020}, parameters have been determined to previously impossible levels of accuracy \citep[e.g.][]{bossini_age_2019,cantat-gaudin_painting_2020}, and many OCs reported before \gaia\ have been ruled out as asterisms \citep{cantat-gaudin_clusters_2020,piatti_assessing_2023,hunt_improving_open_2023}. However, the OC census in the age of \gaia\ remains far from perfect, and one resoundingly large issue stands out that I will attempt to address in this chapter: there is no robust observational criteria or definition for what an OC actually is \citep{hunt_improving_open_2023}.

Following up from the first major catalogue of OCs in the \gaia\ era \citep{cantat-gaudin_gaia_2018}, \citep{cantat-gaudin_clusters_2020} conducted a search for OCs that remained undetected in \gaia\ data, and created a set of empirical observational criteria intended to split dubious objects apart from OCs. This included recommendations that a candidate OC is a clear overdensity with at least $\sim10$ member stars, a colour-magnitude diagram (CMD) that follows a clear isochrone, a median radius $r_{50}$ smaller than 15~pc, and a proper motion dispersion corresponding to an upper limit no greater than 5~km\,s\textsuperscript{-1}. In their work, they used these criteria to show that a number of objects were asterisms.

In \citep{hunt_improving_open_2023} (hereafter Paper 2), we constructed the largest catalogue of OCs to date from a blind search of \gaia\ DR3 data \citep{gaia_collaboration_gaia_2022}. However, we found that the \cite{cantat-gaudin_clusters_2020} observational criteria were too permissive. Many of the objects we detected that passed these criteria were visually much more reminiscent of moving groups (MGs), with sparse, flat distributions that do not resemble the clustered nature of reliable, gravitationally bound OCs. In turn, this creates an awkward situation where our Paper 2 catalogue is challenging to use in many respects, with simple queries of the catalogue returning mostly MGs, particularly within a few hundred pc of the Sun where MGs are most common. It is clear that a more accurate method to distinguish between bound OCs and unbound MGs is required. 

Primarily, it appears that the radius and proper motion constraints presented in \cite{cantat-gaudin_clusters_2020} are the primary area of the OC observational definition that requires improvement. Recalling Eqns.~\ref{eqn:intro:jacobi_radius}~and~\ref{eqn:intro:virial_ratio}, in addition to theories of bound and unbound star clusters discussed in works such as \cite{portegies_zwart_young_2010} and \cite{krause_physics_2020}, one can note that a cluster's mass, radius, and velocity dispersion form a three-way link in defining the dynamical status of a star cluster. Simple inidivudal cuts on radius, dispersion, or even cluster mass should be insufficient to accurately split OCs from MGs.

For example, let us consider a cluster at the upper end of allowed radii from the cuts of \cite{cantat-gaudin_clusters_2020} with $r_{50}=15$~pc. Assuming $r_{50}\sim 2 r_t$ for convenience (which is approximately correct for most OCs), Eqn.~\ref{eqn:intro:jacobi_radius} predicts that in the solar neighbourhood, a cluster of this size would require a total mass of $\sim10^4$~\MSun\ in order to be a bound object with a valid Jacobi radius, a mass so high that it corresponds to some of the largest known clusters in the Milky Way \citep{portegies_zwart_young_2010,cantat-gaudin_milky_2022}. Such a high mass appears unrealistic for the many small, sparse new clusters we detected in the solar neighbourhood in Paper 2. 

On the other hand, the proper motion dispersion upper limit from \cite{cantat-gaudin_clusters_2020} corresponding to 5~\kms\ when converted to velocity dispersions is also likely to be much too high for many clusters. Many of the moving groups we detected in Paper 2 have proper motion dispersions that correspond to velocity dispersions in the range of 1 to 4~\kms. 

Thanfully, theoretical models of star clusters are well studied, particularly for simple \citep{plummer_problem_1911} models for which a wide range of parameters can be derived \citep{portegies_zwart_young_2010}.

% As a thought experiment, one can imagine a reliable cluster such as Melotte~22 (the Pleiades), one of the most nearby and well-studied OCs in the Milky Way. Unsurprisingly, Melotte~22 passes all of the tests outlined in \citep{cantat-gaudin_clusters_2020}. In Paper 2, we detected around 1000 member stars for this cluster. An object with the same distance, radius, and velocity dispersion as Melotte~22 but merely ten member stars (i.e. 1\% of the mass) would have far too little mass in too large a radius and with too much kinetic energy to be gravitationally bound \citep{portegies_zwart_young_2010}. Yet since the 



% PLAN
% - Masses are useful! Dynamics are useful! They're comparable to theory and represent measurements of fundamental (physical) parameters of OCs.
% - BUT: very limited measurements so far of these parameters.
% - Difficult to define OCs robustly without them (Hunt & Reffert 2023)
% - Empirical cuts on parameters insufficient to measure them.
% - MASSES: can be measured more or less two ways: counting stars or tidal parameters
% - discuss pros and cons of each way
% - DYNAMICS: some progress in studying for OCs in MW, but generally for small numbers of clusters only.
% - most clusters found to be supervirial. really?? what about as function of cluster mass, age, etc?
% - finally: masses useful for e.g. analysis of sample completeness. is a more fundamental physical parameter than the observed number of stars
% - give overview of what's in this chapter
% - emphasise that I'll be 


% -------------------------------------
\section{Mass calculations}
\label{sec:dynamics:masses}

% PLAN
% - give a quick overview of what's in this methods section


\subsection{Inference of stellar primary masses}
\label{sec:dynamics:masses:isochrones}

\begin{figure}[t]
    \centering
    \includegraphics[width=0.8\textwidth]{fig/c4/masses_stellar.pdf}
    \caption[TODO]{TODO}
    \label{fig:dynamics:masses:stellar_masses}
 \end{figure}


\subsection{Correction for selection effects}
\label{sec:dynamics:masses:selection}

\begin{figure}[p]
    \centering
    \includegraphics[width=\textwidth]{fig/c4/mass_selection_functions.pdf}
    \caption[TODO]{TODO}
    \label{fig:dynamics:masses:selection_effects}
 \end{figure}


\subsection{Correction for binaries}
\label{sec:dynamics:masses:binaries}


\subsection{Mass function fits}
\label{sec:dynamics:masses:imf_fits}

\begin{figure}[p]
    \centering
    \includegraphics[width=\textwidth]{fig/c4/mass_functions.pdf}
    \caption[TODO]{TODO}
    \label{fig:dynamics:masses:mass_functions}
 \end{figure}


\subsection{Jacobi radius inference}
\label{sec:dynamics:masses:jacobi}

\begin{figure}[t]
    \centering
    \includegraphics[width=\textwidth]{fig/c4/masses_jacobi_radii.pdf}
    \caption[TODO]{TODO}
    \label{fig:dynamics:masses:radii}
 \end{figure}



% -------------------------------------
\section{Velocity dispersion inference}
\label{sec:dynamics:velocities}


\subsection{Gaussian velocity dispersion model}
\label{sec:dynamics:velocities:model}


\subsection{Coordinate frame and radial velocity corrections}
\label{sec:dynamics:velocities:correction}


\subsection{Binary star contamination}
\label{sec:dynamics:velocities:binaries}

\begin{figure}[t]
    \centering
    \includegraphics[width=\textwidth]{fig/c4/dispersion_binaries.pdf}
    \caption[TODO]{TODO}
    \label{fig:dynamics:velocities:binary_contamination}
 \end{figure}


% -------------------------------------
\section{Results}
\label{sec:dynamics:results}


\subsection{Masses}
\label{sec:dynamics:results:masses}

\begin{figure}[t]
    \centering
    \includegraphics[width=\textwidth]{fig/c4/results_mass_comparison.pdf}
    \caption[TODO]{TODO}
    \label{fig:dynamics:results:mass_comparison}
 \end{figure}


\subsection{Jacobi radii}
\label{sec:dynamics:results:radii}

\begin{figure}[t]
    \centering
    \includegraphics[width=\textwidth]{fig/c4/results_p_jac_distribution.pdf}
    \caption[TODO]{TODO}
    \label{fig:dynamics:results:jacobi_radii_distribution}
 \end{figure}


\subsection{Virial ratios}
\label{sec:dynamics:results:virial}

\begin{figure}[t]
    \centering
    \includegraphics[width=\textwidth]{fig/c4/results_virial_vs_mass.pdf}
    \caption[TODO]{TODO}
    \label{fig:dynamics:results:virial_vs_mass}
 \end{figure}

\begin{figure}[t]
    \centering
    \includegraphics[width=\textwidth]{fig/c4/results_q_distribution.pdf}
    \caption[TODO]{TODO}
    \label{fig:dynamics:results:virial_ratio_distribution}
 \end{figure}


\subsection{An updated observational definition of open clusters}
\label{sec:dynamics:results:definition}

\begin{table}[t]

% Define first header
\caption{\label{tab:dynamics:catalogue_results}TODO}

\centering
\begin{tabular}{lccc}
\hline\hline
Type & Criteria & Identifier & Count \\
\hline

OC & $P(r_J) > 0.5$ and $M_J < 50$ \MSun & & TODO \\
- bound OC & $Q_{J,\text{corrected}} \leq 10$ & \texttt{o} & TODO \\
- dissolving OC & $Q_{J,\text{corrected}} \geq 10$ & \texttt{od} & TODO \\
- unbound OC? & $Q_{J,\text{corrected}} \geq 50$ & \texttt{ou} & TODO \\
\hline
MG & $P(r_J) > 0.5$ or $M_J < 50$ & & TODO \\
- without Jacobi component & $P(r_J) < 0.5$ & \texttt{m} & TODO \\
- with Jacobi component & $P(r_J) > 0.5$ & \texttt{mj} & TODO \\
\hline
GC & in \citeme{Vasiliev} & \texttt{g} & TODO \\
\hline

\end{tabular}

% \tablefoot{
% \tablefoottext{a}{}
% }

\end{table}    


% -------------------------------------
\section{Discussion}
\label{sec:dynamics:discussion}


\subsection{Completeness of the \gaia\ DR3 open cluster census}
\label{sec:dynamics:results:completeness}


% -------------------------------------
\section{Conclusions and areas for improvement before publication}
\label{sec:dynamics:conclusion}
